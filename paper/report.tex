\documentclass[10pt,twocolumn]{article} % letterpaper is american!
\usepackage[affil-it]{authblk}

\usepackage[british,UKenglish,USenglish,english,american]{babel}
\usepackage[pdftex]{graphicx}
\usepackage{epstopdf}

\usepackage{amsfonts,amsmath,amsthm,amssymb}

\usepackage{tikz,pgf}
\usetikzlibrary{fit}

\pagestyle{empty}
\setlength{\parindent}{0mm}
\usepackage[letterpaper, margin=1in]{geometry}
%\usepackage{showframe}

\usepackage{multicol}
\usepackage{enumerate}

\usepackage{verbatim}

\usepackage{xspace}
\usepackage{url}
\usepackage{cite}

\usepackage{coffee4}

\usepackage{titlesec}
\titlespacing*{\subsubsection}{0pt}{*0}{*0}
\titlespacing*{\subsection}{0pt}{0pt}{*0}
\titlespacing*{\section}{0pt}{0pt}{*0}

\newcommand{\Bold}{\mathbf}

\setlength{\parskip}{1em}
\setlength{\parindent}{1em}

\title{Cluster Rank Demo Harness}
\date{\today}
\author{Philip Robinson}
\affil{Oregon Health Sciences University}

\begin{document}
\maketitle
\cofeAm{1}{1.0}{0}{5.5cm}{3cm}
\cofeCm{0.9}{1}{180}{0}{0}
\begin{abstract}
  It is often the case that initial query compositions result in frequent restarts as
  the user negotiates with their retrieval system. This is likely a product of unfortunate
  query formulations or choice of ranking algorithm. Our proposed retrieval system
  encourages diversity in displayed documents by introducing an unsupervised clustering
  step before displaying results. The clusters are then presented to the user with their
  documents ranked independent of each group. We do this by clearly seperating the
  retrieval process into the three steps \texttt{relevance}, \texttt{clustering}, and
  \texttt{ranking}, then allow the user to recurse this process on a cluster (rather
  than restarting their query). Additionally, we propose a simple method to
  compare results against varying quality tfidf queries. Our final product is a demo
  harness that abstracts these steps, so that others may easily produce and reproduce
  prototypes against their own corpora.
\end{abstract}

%\section*{General Terms}
%\section*{Keywords}
\section{Introduction}
\section{Implementation Details}
\section{Evaluation Approach}
\section{Expirimental Results}
\section{Limitations}
\section{References}

\end{document}
